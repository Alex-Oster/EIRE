% Autor: Simon May
% Datum: 2016-10-13
% Der Befehl \newcommand kann auch benutzt werden, um „Variablen“ zu definieren:

% Nummer laut Praktikumsheft:
\newcommand*{\varNum}{}
% Name laut Praktikumsheft:
\newcommand*{\varName}{}
% Datum der Durchführung (Format: JJJJ-MM-TT):
\newcommand*{\varDatum}{03. bis 06.09.2018}
% Autoren des Protokolls:
\newcommand*{\varAutor}{David Koke, Alex Oster}
\newcommand*{\varNameA}{David Koke}
\newcommand*{\varNameB}{Alex Oster}
% Nummer der eigenen Gruppe:
\newcommand*{\varGruppe}{}
% E-Mail-Adressen der Autoren (kommagetrennt ohne Leerzeichen!):
\newcommand{\varEmail}{d\_koke01@uni--muenster.de,a\_oste16@uni--muenster.de}
\newcommand{\varEmailA}{d\_koke01@uni--muenster.de}
\newcommand{\varEmailB}{a\_oste16@uni--muenster.de}

%betreuer Name
\newcommand{\varBetreuer}{\normalsize betreut von Jürgen Berkemeier} 
% E-Mail-Adresse anzeigen (true/false):
\newcommand*{\varZeigeEmail}{true}
% Kopfzeile anzeigen (true/false):
\newcommand*{\varZeigeKopfzeile}{true}
% Inhaltsverzeichnis anzeigen (true/false):
\newcommand*{\varZeigeInhaltsverzeichnis}{true}
% Literaturverzeichnis anzeigen (true/false):
\newcommand*{\varZeigeLiteraturverzeichnis}{true}
