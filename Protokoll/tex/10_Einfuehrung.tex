\section{Einführung}

Dieser Bericht dient der Auswertung und Erläuterung der Versuchsergebnisse des Praktikums zum Thema  "Einführung in das rechnergestützte Experimentieren" , durchgeführt im Zeitraum vom 03. bis 06.09.2018, betreut von Dr. Jürgen Berkemeier.
	
\
	
Ziel des Praktikums war die Vermittlung elementarer Kenntnisse zur rechnergestützten Erfassung und Verarbeitung von Messwerten. Dazu wurde das grafische Programmiersystem LabVIEW (Entwickelt von National Instruments, Akronym für "Laboratory Virtual Instrumentation Engineering Workbench"\cite{LV} verwendet, welches wegen der leichten Erlernbarkeit ideal für einen Einstieg in die Thematik geeignet ist.

\

Das Kernelement des Praktikums bestand darin, mit einem Funktionsgenerator erstellte Signale zu digitalisieren und mit Hilfe von LabVIEW in Echtzeit zu verarbeiten. Dazu wurden verschiedene Programme erstellt.

\

Zum Erlernen der grundlegenden Funktionsweise der Programmierumgebung wurde ein Programm erstellt, mit welchem das Produkt zweier Zahlen sowie die Fakultät einer Zahl errechnet werden konnte. Anschließend wurde ein Programm zur Ausgabe eines sinusförmigen Signals geschrieben, welches für spätere Versuchsteile als Unterprogramm verwendet wurde. Im weiteren Verlauf des Praktikums wurde ein Programm erstellt, das mit dem Funktionsgenerator kreierte und mittels einer ADC/DAC-Box erfassten Signale grafisch darstellte und speicherte. Weiterhin wurden Programme zur Fouriertransformation, Modulation und Demodulation der Signale (Verwendung von Amplituden-, Phasen- und Frequenzmodulationsverfahren) erstellt.
